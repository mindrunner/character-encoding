\documentclass[a4paper, 12pt]{scrartcl} % Font size (can be 10pt, 11pt or 12pt) and paper size (remove a4paper for US letter paper)
\usepackage{fancyhdr} % For special header and footer formatting
%\usepackage[T1]{fontenc} % Correct font encoding
%\usepackage[utf8]{inputenc} % Input encoding utf8
\usepackage{xltxtra}
\usepackage{xunicode}
\defaultfontfeatures{Ligatures=TeX}
\usepackage{mathpazo}
\usepackage{mathptmx}
\usepackage{url}
%\usepackage[ngerman]{translator} % using translator for glossaries and other packages
\usepackage{cmbright} % Using computer modern bright font family .. more warnings than Helvetica
%\usepackage[english, mongolian]{babel}
\usepackage{amsmath, amssymb, amsfonts, amsthm}
\usepackage{MnSymbol}
\usepackage{cite}
\usepackage[colorlinks=true, breaklinks=true, linkcolor=HKS41-100, menucolor=darkblue, urlcolor=darkblue]{hyperref} % use hyperref for links with color
\usepackage{graphicx} % Use graphics via \includegraphics
\usepackage{stmaryrd}
%\usepackage{CJK}
\usepackage{CJKutf8}
\usepackage{hyperref}
\usepackage{longtable}
\usepackage{abbrevs}
\usepackage{xkeyval}
\usepackage[toc]{glossaries} % nomain, if you define glossaries in a file, and you use \include{INP-00-glossary}
\usepackage{float}
\usepackage[colorinlistoftodos,prependcaption,linecolor=black,backgroundcolor=green,bordercolor=black]{todonotes}
%\presetkeys{todonotes}{inline,caption=todo}{}
\presetkeys{todonotes}{caption=-}{}
\usepackage{epsfig}
\usepackage{esint}
\usepackage{xcolor} % enhanced color package, replaces color-package
%\usepackage[arrow,matrix,curve]{xy} % xy geometry for drawing
%\usepackage[left=3cm,right=2cm,top=2cm,bottom=2.5cm]{geometry}
%\usepackage{appendix} % appendix support
%\usepackage{colortbl}
\usepackage{listings}
\lstset{basicstyle=\scriptsize, numbers=left, numberstyle=\tiny, stringstyle=\ttfamily, showstringspaces=true, tabsize=1, escapeinside={(*@}{@*)}}
\lstloadlanguages{ML}
\lstset{language=ML}
%\renewcommand{\lstlistingname}{Quelltext}
%\addto\extrasngerman{
%\def\lstlistingautorefname{Quelltext}
%\def\lstnumberautorefname{Zeile}}
\newcommand{\code}[1]{\texttt{#1}}
\usepackage{multicol} % multicolumns
\usepackage{shadethm} % theorems with shadow
%\usepackage{thmbox} % theorems with special boxes .. see http://www.jkrieger.de/tools/latex/mathsimple.html
\usepackage{enumitem} % for better enumerations
%\usepackage{multind} % Indexregister via \index{}, \makeindex{}, \printindex{}
%\usepackage{index}
\usepackage{lastpage} % For counting to last page and probably several other things
\usepackage{ulem} % For correct underlining in breaking lines
\usepackage{tikz}
\usepackage{calc}%    For the \widthof macro
\usetikzlibrary{automata,positioning,calc}
%\usepackage{scalefnt} % \scalefont{2} selects the current font in twice the current 
		      % size. \scalefont{.75} reduces the current font size by three quarters. 
\usepackage{ifthen}
%\usepackage{pdflscape}
\usepackage{xspace} % for xspaces in new commands
%\usepackage{longtable} % longtables for tables over multiple pages

%\usepackage[toc]{glossaries}
%\renewcommand*{\glspostdescription}{}
%\addto\captionsngerman{\renewcommand{\refname}{Quellenverzeichnis}}
%\renewcommand*{\glossaryname}{Glossar}
%\renewcommand*\listfigurename{Abbildungsverzeichnis}
%\renewcommand*{\glossarypreamble}{%
%	  \label{gls:\currentglossary}%
%	}
%\usepackage{pst-barcode}
%\usepackage{auto-pst-pdf}
%\usepackage{blindtext}
%\usepackage{overpic}
\usepackage{tabularx}
%\usepackage{eurosym}

\newboolean{isvc}
\InputIfFileExists{vc.tex}{\setboolean{isvc}{true}}{\setboolean{isvc}{false}}

% ____________________
% |                   |
% |     Commands      |
% |                   |
% |    individual     |
% |     commands      |
% |___________________|
%

\newcommand{\inputown}[1]{\input{content/\projectName/#1}}
\newcommand{\inputcommon}[1]{\input{content/common/#1}}
\newcommand{\inputdata}[1]{\input{content/data/#1}}
\newcommand{\qr}[1]{\begin{pspicture} (1,1)\psbarcode[linecolor=black]{#1}{height=0.8 width=0.8}{qrcode}\end{pspicture}}

%
% Text & Math
%
\renewcommand{\b}[1]{\textbf{#1}} % fetter Text
\renewcommand{\i}[1]{\textit{#1}} % kursiver Text
\newcommand{\ul}[1]{\uline{#1}} % unterstrichen

\newcommand{\N}[0]{\mathbb{N}} % natürliche Zahlen
\newcommand{\Z}[0]{\mathbb{Z}} % ganze Zahlen
\newcommand{\Q}[0]{\mathbb{Q}} % rationale Zahlen
\newcommand{\R}[0]{\mathbb{R}} % reelle Zahlen
%\newcommand{\C}[0]{\mathbb{C}} % komplexe Zahlen
\newcommand{\F}[0]{\mathbb{F}} % Feld (field)
\newcommand{\Ls}[0]{\mathbb{L}} % Lösungsmenge
\newcommand{\A}[0]{\mathbb{A}}

\newcommand{\os}[2]{\overset{#1}{#2}} %
\newcommand{\fr}[2]{\frac{#1}{#2}} % Bruch
\newcommand{\ol}[1]{\overline{#1}} % Überstrich
\newcommand{\ola}[0]{\ol{a}} 
\newcommand{\olb}[0]{\ol{b}}
\newcommand{\olc}[0]{\ol{c}}
\newcommand{\old}[0]{\ol{d}}
\newcommand{\ole}[0]{\ol{e}}
\newcommand{\olf}[0]{\ol{f}}
\newcommand{\oli}[0]{\ol{i}}
\newcommand{\olj}[0]{\ol{j}}
\newcommand{\tl}[1]{\tilde{#1}} % Schlange
\newcommand{\Ra}[0]{\Rightarrow} % doppelter Rechtspfeil
\newcommand{\ra}[0]{\rightarrow} % einfacher Rechtspfeil
\newcommand{\La}[0]{\Leftarrow} % doppelter Linkspfeil
\newcommand{\la}[0]{\leftarrow} % einfacher Linksfpeil
\newcommand{\Lra}[0]{\Leftrightarrow} % doppelter Links-Rechtspfeil
\newcommand{\lra}[0]{\leftrightarrow} % einfacher Links-Rechtspfeil
\newcommand{\mt}[0]{\mapsto} % Pfeil für Abbildungen
\newcommand{\orar}[1]{\overrightarrow{#1}} 
\newcommand{\olar}[1]{\overleftarrow{#1}}
\newcommand{\s}[1]{\langle #1 \rangle} % Skalaprodukt
\newcommand{\ma}[0]{\measuredangle} % Winkelzeichen
\newcommand{\ve}[1]{\left(\begin{array}{c} #1 \end{array} \right)}
\newcommand{\ls}[1]{ ($S$#1)}
\newcommand{\sd}[0]{\backslash}
%\newcommand{\f}[2]{\begin{matrix} #1 \\ #2 \end{matrix}}
\renewcommand{\cupdot}{\makebox[0pt]{\quad \! $\cdot$} \cup}
\newcommand{\class}[1]{\i{#1}}
\newcommand{\interface}[1]{\i{#1}}

\newcommand{\mb}[1]{\mathbb{#1}}

\newcommand{\mf}[1]{\mathfrak{#1}}
\newcommand{\mfA}[0]{\mathfrak{A}}
\newcommand{\mfB}[0]{\mathfrak{B}}
\newcommand{\mfC}[0]{\mathfrak{C}}
\newcommand{\mfD}[0]{\mathfrak{D}}
\newcommand{\mfI}[0]{\mathfrak{I}}
\newcommand{\mfM}[0]{\mathfrak{M}}
\newcommand{\mfP}[0]{\mathfrak{P}}
\newcommand{\mfT}[0]{\mathfrak{T}}
\newcommand{\mfX}[0]{\mathfrak{X}}
\newcommand{\mfE}[0]{\mathfrak{E}}

\newcommand{\mc}[1]{\mathcal{#1}}
\newcommand{\mcA}[0]{\mathcal{A}}
\newcommand{\mcB}[0]{\mathcal{B}}
\newcommand{\mcC}[0]{\mathcal{C}}
\newcommand{\mcD}[0]{\mathcal{D}}
\newcommand{\mcI}[0]{\mathcal{I}}
\newcommand{\mcM}[0]{\mathcal{M}}
\newcommand{\mcP}[0]{\mathcal{P}}
\newcommand{\mcT}[0]{\mathcal{T}}
\newcommand{\mcX}[0]{\mathcal{X}}
\newcommand{\mcE}[0]{\mathcal{E}}

%
% Colors & Font Formatting
%
\newcommand{\changefont}[3]{\fontfamily{#1} \fontseries{#2} \fontshape{#3} \selectfont} % change format easier

    \definecolor{HKS41-100}{cmyk}{1.0, 0.7, 0.1, 0.5}
    \definecolor{HKS41-90}{cmyk}{0.9, 0.63, 0.09, 0.45}
    \definecolor{HKS41-80}{cmyk}{0.8, 0.56, 0.08, 0.4}
    \definecolor{HKS41-70}{cmyk}{0.7, 0.49, 0.07, 0.35}
    \definecolor{HKS41-60}{cmyk}{0.6, 0.42, 0.06, 0.3}
    \definecolor{HKS41-50}{cmyk}{0.5, 0.35, 0.05, 0.25}
    \definecolor{HKS41-40}{cmyk}{0.4, 0.28, 0.04, 0.2}
    \definecolor{HKS41-30}{cmyk}{0.3, 0.21, 0.03, 0.15}
    \definecolor{HKS41-20}{cmyk}{0.2, 0.14, 0.02, 0.1}
    \definecolor{HKS41-10}{cmyk}{0.1, 0.07, 0.01, 0.05}
    \definecolor{HKS44-100}{cmyk}{1.0, 0.5, 0.0, 0.0}
    \definecolor{HKS44-90}{cmyk}{0.9, 0.45, 0.0, 0.0}
    \definecolor{HKS44-80}{cmyk}{0.8, 0.4, 0.0, 0.0}
    \definecolor{HKS44-70}{cmyk}{0.7, 0.35, 0.0, 0.0}
    \definecolor{HKS44-60}{cmyk}{0.6, 0.3, 0.0, 0.0}
    \definecolor{HKS44-50}{cmyk}{0.5, 0.25, 0.0, 0.0}
    \definecolor{HKS44-40}{cmyk}{0.4, 0.2, 0.0, 0.0}
    \definecolor{HKS44-30}{cmyk}{0.3, 0.15, 0.0, 0.0}
    \definecolor{HKS44-20}{cmyk}{0.2, 0.1, 0.0, 0.0}
    \definecolor{HKS44-10}{cmyk}{0.1, 0.05, 0.0, 0.0}
    \definecolor{HKS36-100}{cmyk}{0.8, 0.9, 0.0, 0.0}
    \definecolor{HKS36-90}{cmyk}{0.72, 0.81, 0.0, 0.0}
    \definecolor{HKS36-80}{cmyk}{0.64, 0.72, 0.0, 0.0}
    \definecolor{HKS36-70}{cmyk}{0.56, 0.63, 0.0, 0.0}
    \definecolor{HKS36-60}{cmyk}{0.48, 0.54, 0.0, 0.0}
    \definecolor{HKS36-50}{cmyk}{0.4, 0.45, 0.0, 0.0}
    \definecolor{HKS36-40}{cmyk}{0.32, 0.36, 0.0, 0.0}
    \definecolor{HKS36-30}{cmyk}{0.24, 0.27, 0.0, 0.0}
    \definecolor{HKS36-20}{cmyk}{0.16, 0.18, 0.0, 0.0}
    \definecolor{HKS36-10}{cmyk}{0.08, 0.09, 0.0, 0.0}
    \definecolor{HKS33-100}{cmyk}{0.5, 1.0, 0.0, 0.0}
    \definecolor{HKS33-90}{cmyk}{0.45, 0.9, 0.0, 0.0}
    \definecolor{HKS33-80}{cmyk}{0.4, 0.8, 0.0, 0.0}
    \definecolor{HKS33-70}{cmyk}{0.35, 0.7, 0.0, 0.0}
    \definecolor{HKS33-60}{cmyk}{0.3, 0.6, 0.0, 0.0}
    \definecolor{HKS33-50}{cmyk}{0.25, 0.5, 0.0, 0.0}
    \definecolor{HKS33-40}{cmyk}{0.2, 0.4, 0.0, 0.0}
    \definecolor{HKS33-30}{cmyk}{0.15, 0.3, 0.0, 0.0}
    \definecolor{HKS33-20}{cmyk}{0.1, 0.2, 0.0, 0.0}
    \definecolor{HKS33-10}{cmyk}{0.05, 0.1, 0.0, 0.0}
    \definecolor{HKS57-100}{cmyk}{1.0, 0.0, 0.9, 0.2}
    \definecolor{HKS57-90}{cmyk}{0.9, 0.0, 0.81, 0.18}
    \definecolor{HKS57-80}{cmyk}{0.8, 0.0, 0.72, 0.16}
    \definecolor{HKS57-70}{cmyk}{0.7, 0.0, 0.63, 0.14}
    \definecolor{HKS57-60}{cmyk}{0.6, 0.0, 0.54, 0.12}
    \definecolor{HKS57-50}{cmyk}{0.5, 0.0, 0.45, 0.1}
    \definecolor{HKS57-40}{cmyk}{0.4, 0.0, 0.36, 0.08}
    \definecolor{HKS57-30}{cmyk}{0.3, 0.0, 0.27, 0.06}
    \definecolor{HKS57-20}{cmyk}{0.2, 0.0, 0.18, 0.04}
    \definecolor{HKS57-10}{cmyk}{0.1, 0.0, 0.09, 0.02}
    \definecolor{HKS65-100}{cmyk}{0.65, 0.0, 1.0, 0.0}
    \definecolor{HKS65-90}{cmyk}{0.585, 0.0, 0.9, 0.0}
    \definecolor{HKS65-80}{cmyk}{0.52, 0.0, 0.8, 0.0}
    \definecolor{HKS65-70}{cmyk}{0.455, 0.0, 0.7, 0.0}
    \definecolor{HKS65-60}{cmyk}{0.39, 0.0, 0.6, 0.0}
    \definecolor{HKS65-50}{cmyk}{0.325, 0.0, 0.5, 0.0}
    \definecolor{HKS65-40}{cmyk}{0.26, 0.0, 0.4, 0.0}
    \definecolor{HKS65-30}{cmyk}{0.195, 0.0, 0.3, 0.0}
    \definecolor{HKS65-20}{cmyk}{0.13, 0.0, 0.2, 0.0}
    \definecolor{HKS65-10}{cmyk}{0.065, 0.0, 0.1, 0.0}
    \definecolor{HKS07-100}{cmyk}{0.0, 0.6, 1.0, 0.0}
    \definecolor{HKS07-90}{cmyk}{0.0, 0.54, 0.9, 0.0}
    \definecolor{HKS07-80}{cmyk}{0.0, 0.48, 0.8, 0.0}
    \definecolor{HKS07-70}{cmyk}{0.0, 0.42, 0.7, 0.0}
    \definecolor{HKS07-60}{cmyk}{0.0, 0.36, 0.6, 0.0}
    \definecolor{HKS07-50}{cmyk}{0.0, 0.3, 0.5, 0.0}
    \definecolor{HKS07-40}{cmyk}{0.0, 0.24, 0.4, 0.0}
    \definecolor{HKS07-30}{cmyk}{0.0, 0.18, 0.3, 0.0}
    \definecolor{HKS07-20}{cmyk}{0.0, 0.12, 0.2, 0.0}
    \definecolor{HKS07-10}{cmyk}{0.0, 0.06, 0.1, 0.0}
    % Benannte Farben
    \definecolor{white}{gray}{1.00}
    \definecolor{black}{gray}{0.00}
    \definecolor{skyblue}{cmyk}{0.4, 0.2, 0.0, 0.0}             % HKS44-40
    \definecolor{blue}{cmyk}{1.0, 0.5, 0.0, 0.0}                % HKS44-100
    \definecolor{lightblue}{cmyk}{0.7, 0.35, 0.0, 0.0}          % HKS44-70
    \definecolor{darkblue}{rgb}{0.04, 0.16, 0.32}               % 
    \definecolor{extradarkblue}{cmyk}{1.00, 0.70, 0.10, 0.50}   % HKS41-100
    \definecolor{darkgreen}{cmyk}{1.0, 0.0, 0.9, 0.2}           % HKS57-100
    \definecolor{green}{cmyk}{0.65, 0.0, 1.0, 0.0}              % HKS65-100
    \definecolor{purple}{cmyk}{0.5, 1.0, 0.0, 0.0}              % HKS33-100
    \definecolor{indigo}{cmyk}{0.8, 0.9, 0.0, 0.0}              % HKS36-100
    \definecolor{gray}{gray}{0.59}
    \definecolor{darkgray}{gray}{0.50}
    \definecolor{darkcyan}{cmyk}{0.87, 0.4, 0.4, 0.0}
    \definecolor{cyan}{cmyk}{0.78, 0.19, 0.01, 0.0}
    \definecolor{lightcyan}{cmyk}{0.39, 0.095, 0.005, 0.0}
    \definecolor{extralightcyan}{cmyk}{0.16, 0.1, 0.0, 0.0}


%
% Miscelauneous
%
\newcommand{\dividingRule}[1]{\colorlet{tempCurrentColor}{.}\color{#1}\rule{\textwidth}{.2pt}\color{tempCurrentColor}}

% Highlighting. Usage: \highlight{text}
\definecolor{highlightColor}{cmyk}{1.00, 0.70, 0.10, 0.50}
\newcommand{\highlight}[1]{\colorlet{tempCurrentColor}{.}\color{highlightColor}#1\color{tempCurrentColor}\xspace}

% Marking todo's. Usage: \todo{my open task}
%\definecolor{todoColor}{cmyk}{0, 1.00, 1.00, 0.00}
%\newcommand{\todo}[1]{\colorlet{tempCurrentColor}{.}\color{todoColor}\i{/* #1 */}\color{tempCurrentColor}\xspace}

% Quote environment with small text in italic and 75% lighter using xcolor
\expandafter\def\expandafter\quote\expandafter{\quote\color{.!50!HKS44-40}\textbf{``}\it\small}

% Mehrere eigene Umgebungen für schnellere Formatierung
\newenvironment{enumroman}
	{\begin{enumerate}[label=\roman*),labelindent=\parindent]\setlength{\labelsep}{1em}}
	{\end{enumerate}}
\newenvironment{enum}
	{\begin{itemize}\setlength{\labelsep}{1em}}
	{\end{itemize}}
\newenvironment{enumalpha}
	{\begin{enumerate}[label=\alph*),labelindent=\parindent]\setlength{\labelsep}{1em}}
	{\end{enumerate}}
\newenvironment{enumarabic}
	{\begin{enumerate}[label=\arabic*),labelindent=\parindent]\setlength{\labelsep}{1em}}
	{\end{enumerate}}
\newenvironment{enumequiv}
	{\begin{enumerate}[label=(\roman*),align=right,labelindent=\parindent]\setlength{\labelsep}{1em}}
	{\end{enumerate}}


% matrix-Umgebungen um Ausrichtung und vlines erweitert
%\makeatletter
%\renewcommand*\env@matrix[1][*\c@MaxMatrixCols c]{%
%  \hskip -\arraycolsep
%  \let\@ifnextchar\new@ifnextchar
%  \array{#1}}
%\makeatother
%
%% Beweisumgebung
%\renewcommand*\proofname{Beweis}
%\makeatletter\renewenvironment{proof}[1][\proofname]{
%	\colorlet{@tempCurrentColor}{.}\color{.!70}\par\pushQED{\qed}\normalfont\topsep6\p@\@plus6\p@\relax\trivlist\item[\hskip\labelsep\bfseries #1\@addpunct{:}]~\newline\ignorespaces
%}{
%  \popQED\endtrivlist\@endpefalse\color{@tempCurrentColor}
%}\makeatother
%
%% Environments for terms with indented description text following
%\makeatletter\newenvironment{term}[1]{
%	\par\hangindent=3em\normalfont\topsep6\p@\@plus6\p@\relax\trivlist\item[\hskip\labelsep\bfseries #1]
%}{
%	\endtrivlist
%}\makeatother
%
%
%\makeatletter
%\patchcmd{\Gin@ii}
%  {\begingroup}
%  {\begingroup\renewcommand{\@latex@error}[2]{\includegraphics[width=0.2\textwidth]{images/placeholder}}}
%  {}
%  {}
%	\makeatother

%
% Fine little colored boxes
%
%

% [jjs]ColorBox - using xcolor
% \begin{jjsColorBox}[optBgColor] .. \end{jjsColorBox}
\definecolor{jjsColorBoxBorderColor}{cmyk}{0,0,0,1}
\makeatletter\newenvironment{jjsColorBox}[1][white]{
	\colorlet{@tempInfoColor}{#1}\colorlet{@tempInfoColor2}{jjsColorBoxBorderColor}\par\vspace{\baselineskip}\noindent\begin{lrbox}{\@tempboxa}\begin{minipage}[b]{\linewidth}
}{
	\end{minipage}\end{lrbox}\setlength{\fboxsep}{3pt}\fcolorbox{@tempInfoColor2}{@tempInfoColor}{\usebox{\@tempboxa}}\par\noindent
}\makeatother

% Zusatz-Info Boxen
%\makeatletter\newenvironment{info}{\begin{jjsColorBox}[HKS57-20]\color{.!90}
%}{\end{jjsColorBox}}\makeatother


%
% Sätze, Definitionen, Theoreme
%
\theoremstyle{definition}
\newshadetheorem{theorems}{Theorem}
\newshadetheorem{corollaries}{Korollar}
\newshadetheorem{lemmas}{Lemma}
\newshadetheorem{axioms}{Axiom}
\newshadetheorem{sentences}{Satz}
\newshadetheorem{definitions}{Definition}
\newshadetheorem{cites}{Zitat}
\newshadetheorem{example}{Beispiel}

\newenvironment{theorem}[1][]{
	\colorlet{shadethmcolor}{HKS44-20}
	\colorlet{shaderulecolor}{HKS41-80}
	\setlength{\shadeboxrule}{.5pt}
	\begin{theorems}[#1]\hspace*{1mm}
}{\end{theorems}}

\newenvironment{definition}[1][]{
	\colorlet{shadethmcolor}{HKS44-20}
	\colorlet{shaderulecolor}{HKS41-80}
	\setlength{\shadeboxrule}{.5pt}
	\begin{definitions}[#1]\hspace*{1mm}
}{\end{definitions}}

\newenvironment{sentence}[1][]{
	\colorlet{shadethmcolor}{HKS44-20}
	\colorlet{shaderulecolor}{HKS41-80}
	\setlength{\shadeboxrule}{.5pt}
	\begin{sentences}[#1]\hspace*{1mm}
}{\end{sentences}}

\newenvironment{corollary}[1][]{
	\colorlet{shadethmcolor}{HKS44-20}
	\colorlet{shaderulecolor}{HKS41-80}
	\setlength{\shadeboxrule}{.5pt}
	\begin{corollaries}[#1]\hspace*{1mm}
}{\end{corollaries}}

\newenvironment{lemma}[1][]{
	\colorlet{shadethmcolor}{HKS44-20}
	\colorlet{shaderulecolor}{HKS41-80}
	\setlength{\shadeboxrule}{.5pt}
	\begin{lemmas}[#1]\hspace*{1mm}
}{\end{lemmas}}

\newenvironment{axiom}[1][]{
	\colorlet{shadethmcolor}{HKS44-20}
	\colorlet{shaderulecolor}{HKS41-80}
	\setlength{\shadeboxrule}{0.7pt}
	\begin{axioms}[#1]\hspace*{1mm}
}{\end{axioms}}

\newenvironment{citebox}[1][]{
	\colorlet{shadethmcolor}{HKS44-20}
	\colorlet{shaderulecolor}{HKS41-80}
	\setlength{\shadeboxrule}{0.7pt}
	\begin{cites}[#1]\hspace*{1mm}
}{\end{cites}}

\newenvironment{examplebox}[1][]{
	\colorlet{shadethmcolor}{HKS44-20}
	\colorlet{shaderulecolor}{HKS41-80}
	\setlength{\shadeboxrule}{0.7pt}
	\begin{example}[#1]\hspace*{1mm}
}{\end{example}}


\newcommand{\tikzmark}[1]{\tikz[overlay,remember picture] \node (#1) {};}

%\newcommand{\AND}[0]{\wedge}
%\newcommand{\OR}[0]{\vee}
\providecommand{\abs}[1]{\lvert#1\rvert}
\providecommand{\norm}[1]{\lVert#1\rVert}

\usepackage{xparse}

\NewDocumentCommand{\DrawBox}{s O{}}{%
    \tikz[overlay,remember picture]{
    \IfBooleanTF{#1}{%
        \coordinate (RightPoint) at ($(left |- right)+(\linewidth-\labelsep-\labelwidth,0.0)$);
    }{%
        \coordinate (RightPoint) at (right.east);
    }%
    \draw[red,#2]
      ($(left)+(-0.2em,0.9em)$) rectangle
      ($(RightPoint)+(0.2em,-0.3em)$);}
}

\NewDocumentCommand{\DrawBoxWide}{s O{}}{%
    \tikz[overlay,remember picture]{
    \IfBooleanTF{#1}{%
        \coordinate (RightPoint) at ($(left |- right)+(\linewidth-\labelsep-\labelwidth,0.0)$);
    }{%
        \coordinate (RightPoint) at (right.east);
    }%
    \draw[red,#2]
      ($(left)+(-\labelwidth,0.9em)$) rectangle
      ($(RightPoint)+(0.2em,-0.3em)$);}
}


\newcommand{\ltlNext}[1]{\mathcal{X}#1}
\newcommand{\ltlGlobal}[1]{\mathcal{G}#1}
\newcommand{\ltlFuture}[1]{\mathcal{F}#1}
\newcommand{\ltlUntil}[2]{#1\mathcal{U}#2}
\newcommand{\ltlWeak}[2]{#1\mathcal{W}#2}
\newcommand{\ltlRelease}[2]{#1\mathcal{R}#2}
\newcommand{\ctlE}[0]{\mathcal{E}}
\newcommand{\ctlA}[0]{\mathcal{A}}

\usepackage{lmodern}
\input{glossary.glo}
\makeglossaries
%\usepackage[xindy]{imakeidx}
%\makeindex
%\usepackage{fontspec}
\usepackage[english]{selnolig}
\newfontfamily\mongolian[Path=fonts/noto/]{NotoSansMongolian-Regular.ttf}
\newfontfamily\chineset[Path=fonts/noto-cjk/]{NotoSansHant-Regular.otf}
\newfontfamily\chineses[Path=fonts/noto-cjk/]{NotoSansHans-Regular.otf}
\newfontfamily\myanmar[Path=fonts/noto/]{NotoSansMyanmar-Regular.ttf}
\newfontfamily\gujarati[Path=fonts/noto/]{NotoSansGujarati-Regular.ttf}
\newfontfamily\droid[]{DroidSansFallback}
\newfontfamily\droidthai[]{DroidSansThai}
\newfontfamily\arial{Arial}
\newfontfamily\dejavu{DejaVuSans}
\linespread{1.15} % Change line spacing here, Palatino benefits from a slight increase by default
\makeatletter
\renewcommand{\maketitle}{ % Customize the title - do not edit title and author name here, see the TITLE block below
  \begin{flushright} % Right align
    {\LARGE\@title}\\ % Increase the font size of the title
    \vspace{50pt} % Some vertical space between the title and author name
    {\large\@author} % Author name
    \\\@date % Date
    \vspace{40pt} % Some vertical space between the author block and abstract
  \end{flushright}
}
\title{\textbf{Character encoding}\\ % Title
a fundamental part of software internationalisation} % Subtitle
\author{\textsc{Lukas Elsner} % Author
\\{\textit{Queensland University of Technology}}} % Institution
\date{\today} % Date
\begin{document}
\maketitle % Print the title section
\thispagestyle{empty}
%\renewcommand{\abstractname}{Summary} % Uncomment to change the name of the abstract to something else
\vspace{10em}
\begin{abstract}
  Software constructed for the global market must be aware of regional
  differences, such as different languages, culture dependent formatting and
  customised user interfaces. Without a character encoding technique, which can
  handle all characters used worldwide, every software needs to add its own
  support for handling different character sets. Since this is a non-trivial
  task, the overhead to globalise and internationalise a software would be a
  huge impact on development time and costs. Therefore, a uniform code that can
  interpret all known characters is needed as the foundation of all
  internationalisation tasks in software development. The current solution,
  Unicode, solves this problem for the most part. Although not all problems were
  resolved by it.
\end{abstract}
\newpage
%\listoftodos
\tableofcontents
\newpage
%\hspace*{3,6mm}\textit{Keywords:} lorem , ipsum , dolor , sit amet , lectus % Keywords
%\vspace{30pt} % Some vertical space between the abstract and first section
%----------------------------------------------------------------------------------------
%	ESSAY BODY
%----------------------------------------------------------------------------------------

\section{Introduction}

Computers are being used worldwide and most of the software is not constructed
for a special country or region. Therefore, software developers should take
this into account and develop software in a globalised and localised way.
Globalised software must be independent of different languages and formats,
such as date and number format. Globalised software is the basis for creating a
world-ready application. Beginning with a globalised framework, software can
easily be localised for every culture and language. \\
A fundamental task of most software is processing text. A huge amount of
software on the market communicates with its user by text input and output.
Therefore the software's internal representation of strings must be prepared
for many different characters and should be able to handle all known character
sets in a unified way. A good approach to fulfill this requirement is for the
complete chain of involved software, such as operating system, integrated
development environment and compiler, to handle strings in a compatible way. \\
The history of encoding characters is long and it seems that every time a new
approach was invented, new problems arose. Most modern software systems today
use Unicode, which is a unified mapping between every character and a unique
number. This mapping makes it possible for every software to distinguish
between different characters without any misunderstanding. Still, even Unicode
has not yet solved all problems, such as reliable encoding detection, as the
paper will explore.

\section{History of character encoding}

The history of character encoding begins long before the history of computers.
The first known code-based system used for long-distance transmission of
information was invented by the Greeks around 350 {\smaller{\acrshort{BCE}}}.
Many years later, several other encoding methods were invented to transmit
data. Perhaps the most famous code is the Morse code, created by Alfred Vail
and Samuel Morse for their electrical telegraph system in 1837, which was the
first internationally used encoding system \cite{haralambous2007fonts}. Still,
even Morse code does not fulfill all of the requirements needed to encode all
known characters. At the beginning of the twentieth century, long-distance
communication was conducted using telegraphic systems. In 1931, the \gls{CCITT}
standardised the \glslink{CCITT}{CCITT \#2}, an international 58-character
shifted 5-bit code \cite{haralambous2007fonts}. \\
Around 1950, when the first stored-program computer was invented, the need for
a suitable character encoding scheme was born. Computers can only handle
numbers in binary representation, but humanity requires them to do more than
number crunching. One of the main input/output tasks that people require is
the processing of text. Towards the end of the 1950s, two different approaches
were developed independently to define a standard in character representation
on computers that would allow text processing.
\begin{itemize}
  \item{\gls{EBCDIC}} \\
    IBM developed the 8-bit \gls{EBCDIC}, which was used mainly on IBM's
    mainframe and midrange computers as well as on other non-IBM platforms.
    \gls{EBCDIC} was derived from \gls{BCD}, an encoding system mainly used on
    \glspl{punchcard}.
  \item{\gls{ASCII}} \\
    The \gls{ANSI} defined a character encoding based on previously-used
    telegraph codes. In 1963, the first \gls{ASCII} standard, \glslink{ASCII}{ASCII-1963},
    was defined and used commercially \cite{brandle1999ascii}. \gls{ASCII} is a
    7-bit character encoding, designed for storing and transmitting English
    text. It includes 33 non-printing control and 95 printable characters. In
    1967, \gls{ASCII} was extended to lowercase characters and eventually
    became the ISO 646 standard in 1983 \cite{haralambous2007fonts}.
    Table~\ref{table:asciitable} shows the current mapping of plain
    \gls{ASCII}. \\
    \begin{table}[H]
      \begin{center}
        \scalebox{0.6}{
          \begin{tabular}{|c|c|c|c||c|c|c|c||c|c|c|c||c|c|c|c|}
            \hline
            Dec & Hex & Oct & Char & Dec & Hex & Oct & Char & Dec & Hex & Oct & Char & Dec & Hex & Oct & Char \\
            \hline
            0 & 0x00 & 000 & NUL & 32 & 0x20 & 040 & SP & 64 & 0x40 & 100 & @ & 96 & 0x60 & 140 & ` \\
            1 & 0x01 & 001 & SOH & 33 & 0x21 & 041 & !  & 65 & 0x41 & 101 & A & 97 & 0x61 & 141 & a \\
            2 & 0x02 & 002 & STX & 34 & 0x22 & 042 & "' & 66 & 0x42 & 102 & B & 98 & 0x62 & 142 & b \\
            3 & 0x03 & 003 & ETX & 35 & 0x23 & 043 & \# & 67 & 0x43 & 103 & C & 99 & 0x63 & 143 & c \\
            4 & 0x04 & 004 & EOT & 36 & 0x24 & 044 & \$ & 68 & 0x44 & 104 & D & 100 & 0x64 & 144 & d \\
            5 & 0x05 & 005 & ENQ & 37 & 0x25 & 045 & \% & 69 & 0x45 & 105 & E & 101 & 0x65 & 145 & e \\
            6 & 0x06 & 006 & ACK & 38 & 0x26 & 046 & \& & 70 & 0x46 & 106 & F & 102 & 0x66 & 146 & f \\
            7 & 0x07 & 007 & BEL & 39 & 0x27 & 047 & ' & 71 & 0x47 & 107 & G & 103 & 0x67 & 147 & g \\
            8 & 0x08 & 010 & BS & 40 & 0x28 & 050 & ( & 72 & 0x48 & 110 & H & 104 & 0x68 & 150 & h \\
            9 & 0x09 & 011 & TAB & 41 & 0x29 & 051 &  ) & 73 & 0x49 & 111 & I & 105 & 0x69 & 151 & i \\
            10 & 0x0A & 012 & LF & 42 & 0x2A & 052 & * & 74 & 0x4A & 112 & J & 106 & 0x6A & 152 & j \\
            11 & 0x0B & 013 & VT & 43 & 0x2B & 053 & + & 75 & 0x4B & 113 & K & 107 & 0x6B & 153 & k \\
            12 & 0x0C & 014 & FF & 44 & 0x2C & 054 & , & 76 & 0x4C & 114 & L & 108 & 0x6C & 154 & l \\
            13 & 0x0D & 015 & CR & 45 & 0x2D & 055 & - & 77 & 0x4D & 115 & M & 109 & 0x6D & 155 & m \\
            14 & 0x0E & 016 & SO & 46 & 0x2E & 056 & . & 78 & 0x4E & 116 & N & 110 & 0x6E & 156 & n \\ 
            15 & 0x0F & 017 & SI & 47 & 0x2F & 057 & / & 79 & 0x4F & 117 & O & 111 & 0x6F & 157 & o \\
            16 & 0x10 & 020 & DLE & 48 & 0x30 & 060 & 0 & 80 & 0x50 & 120 & P & 112 & 0x70 & 160 & p \\
            17 & 0x11 & 021 & DC1 & 49 & 0x31 & 061 & 1 & 81 & 0x51 & 121 & Q & 113 & 0x71 & 161 & q \\
            18 & 0x12 & 022 & DC2 & 50 & 0x32 & 062 & 2 & 82 & 0x52 & 122 & R & 114 & 0x72 & 162 & r \\
            19 & 0x13 & 023 & DC3 & 51 & 0x33 & 063 & 3 & 83 & 0x53 & 123 & S & 115 & 0x73 & 163 & s \\
            20 & 0x14 & 024 & DC4 & 52 & 0x34 & 064 & 4 & 84 & 0x54 & 124 & T & 116 & 0x74 & 164 & t \\
            21 & 0x15 & 025 & NAK & 53 & 0x35 & 065 & 5 & 85 & 0x55 & 125 & U & 117 & 0x75 & 165 & u \\
            22 & 0x16 & 026 & SYN & 54 & 0x36 & 066 & 6 & 86 & 0x56 & 126 & V & 118 & 0x76 & 166 & v \\
            23 & 0x17 & 027 & ETB & 55 & 0x37 & 067 & 7 & 87 & 0x57 & 127 & W & 119 & 0x77 & 167 & w \\
            24 & 0x18 & 030 & CAN & 56 & 0x38 & 070 & 8 & 88 & 0x58 & 130 & X & 120 & 0x78 & 170 & x \\
            25 & 0x19 & 031 & EM & 57 & 0x39 & 071 & 9 & 89 & 0x59 & 131 & Y & 121 & 0x79 & 171 & y \\
            26 & 0x1A & 032 & SUB & 58 & 0x3A & 072 & : & 90 & 0x5A & 132 & Z & 122 & 0x7A & 172 & z \\
            27 & 0x1B & 033 & ESC & 59 & 0x3B & 073 & ; & 91 & 0x5B & 133 & [ & 123 & 0x7B & 173 & \{ \\
            28 & 0x1C & 034 & FS & 60 & 0x3C & 074 & "< & 92 & 0x5C & 134 & $\backslash$ & 124 & 0x7C & 174 & $\mid$ \\
        29 & 0x1D & 035 & GS & 61 & 0x3D & 075 & = & 93 & 0x5D & 135 & ] & 125 & 0x7D & 175 & \} \\
            30 & 0x1E & 036 & RS & 62 & 0x3E & 076 & "> & 94 & 0x5E & 136 & \^{} & 126 & 0x7E & 176 & "~ \\
            31 & 0x1F & 037 & US & 63 & 0x3F & 077 & ? & 95 & 0x5F & 137 & \_ & 127 & 0x7F & 177 & DEL \\
            \hline
        \end{tabular}}
      \end{center}
      \caption{\gls{ASCII} table \cite{latexasciitable}}
      \label{table:asciitable}
    \end{table}
\end{itemize}
In many countries, the original \gls{ASCII} encoding was modified to fulfill
particular needs. In order to create a localised version of the original
\gls{ASCII}, certain characters were replaced. For example, the currency
symbol, which differs between countries, was substituted with a character that
represents the local currency. Some examples of modified \gls{ASCII} are NF
Z62010 in France, BS 4730 in United Kingdom and JIS C-6220 in Japan
\cite{haralambous2007fonts}. \\
Besides creating those modifications for a particular country, there were so
called ``Extended-\gls{ASCII}'' character sets, which used the eighth bit and
could take advantage of 128 more characters. A widespread version of this is
the codepage \gls{cp-125x} from Microsoft, which later became a slightly different
version known as the \glslink{ISO-8859}{ISO-8859-x} standard \cite{korpela2006unicode}. \\
An additional approach in the direction of internationalised software was the
ability to use multiple code pages in one document. The ISO 2022 standard was
developed as a general framework to switch between code pages using special
control codes. This complex standard, which is not compatible with the Windows
code pages, has never been implemented completely and was only used for East
Asian languages \cite{korpela2006unicode}. The ISO 2002 standard is a stateful
encoding system, which means that a control code within the document can change
the interpretation of a character code. One disadvantage of switching between
code pages within one document is that the document must be read from the
beginning. Random access at a particular position is not possible and may
lead to misinterpretation of character codes. Therefore, a new and more uniform
approach was needed. \\
The development of a universal character set, called Unicode, can be traced
to late 1987, when engineers of Apple and Xerox began discussing this new
encoding approach. Joe Becker, Lee Collins and Mark Davis started to
investigate a universal character encoding system. They aimed for three main goals:
\begin{itemize}
  \item{\textbf{Universal}} \\
    The needs of real world languages must be addressed
  \item{\textbf{Uniform}} \\
    Fixed-width codes, to preserve efficient access
  \item{\textbf{Unique}} \\
    One bit sequence must only have one interpretation into a character code
\end{itemize}
In fall 1988, the work on the first Unicode database began The result was the
first Unicode database between 1988 and 1990, which mainly preserved the
mappings of \gls{ASCII} and \gls{ISO-8859} standards, to maintain backward
compatibility. \cite{summarynarrative}.

\section{Encoding characters today}

Almost every computer today uses Unicode encoding as its default character
encoding system. Unicode is a unique mapping between a character and a positive
integer. There are a couple of prerequisites to understand how
Unicode works. One of the most important things to know is that Unicode itself
does not specify how encoding has to be done. Another important factor, which
should not to be underestimated, is the definition of the word ``character''.
This chapter will define ``character'' and give a brief introduction into the
mappings of Unicode as well as the multiple possibilities to encode Unicode
into a binary representation.

\subsection{Characters}

When talking about characters many people might think of a particular letter in
their language, which is correct, but a character can be even more than a
simple `a'. It can also be an accented `a', like `\'a' or `\"a'. A great number
of accented characters are used in languages using the Latin alphabet, so a
unique representation must be designated for each one. A character can also be
a symbol which many people have not seen before in their life. For example the
Mongolian letter `todo pa' ({\mongolian ᡌ}) or the Thai character `no nen'
({\droidthai ณ}). Characters can also be ligatures, which are composed letters
containing two or more characters. One well known ligature is the Latin `fi'.
Another one is the Latin lowercase `i\breaklig j' which can be explicitly typed
as one character `ij'. Depending of the font, there might not be any difference
in the rendering.  Using the definition of character given above, there are
thousands of characters that must be considered by a unified encoding system.
This system must also be able to add more characters in the future.

\subsection{Unicode}

Unicode is divided in 17 so called \glspl{plane}. Each \gls{plane} can hold up
to 65,536 different characters. Thus, Unicode can address 1,114,112 different
characters and its address space length is 21-bit. Each of these addresses is
called a \gls{codepoint}. In June 2014, the Unicode consortium released the
Unicode standard 7.0, which addresses a total of 123 scripts with 113,021
characters. \cite{unicode7release} To refer to a particular Unicode character, the general
writing of `U+' followed by the \gls{codepoint}s hexadecimal number was
established. This hexadecimal number is the concatenation of \gls{plane} number
and the address within the \gls{plane}, where the \gls{plane} number can be
ignored when zero. For example, to reference character 3603 in \gls{plane} 0,
we can just write `U+0e13'.

\subsection{\acrlong{UTF}}

The \gls{UTF} refers to several types of character
encodings which are used to transform Unicode characters into binary
representations. If a computer encodes characters in Unicode, it needs to use
one of Unicode's character encoding formats, \glslink{UTF}{UTF-8},
\glslink{UTF}{UTF-16} or \glslink{UTF}{UTF-32}. While \glslink{UTF}{UTF-16}
and \glslink{UTF}{UTF-32} can be subdivided into the two byte orders `\glslink{Endianness}{big endian}' and `\glslink{Endianness}{little endian}`, \glslink{UTF}{UTF-8}, by
design, will always have the same byte order. \glslink{UTF}{UTF-8} and
\glslink{UTF}{UTF-16} are variable-width encodings which means, that one
particular character will be represented as one to four bytes in
\glslink{UTF}{UTF-8} and as two or four bytes in \glslink{UTF}{UTF-16}.
\glslink{UTF}{UTF-32} has a fixed-width length of four bytes.
\glslink{UTF}{UTF-8} is very widespread in the World Wide Web and is the default
encoding in most of common Linux distributions. Microsoft Windows uses
\glslink{UTF}{UTF-16} for its internal data representation.
\glslink{UTF}{UTF-32} is not commonly used, because of its huge data overhead.
\glslink{UTF}{UTF-32} is usually only used when it is necessary to quickly
and randomly access the n-th \gls{codepoint} of a document, because this is only
possible with a fixed-width format. Table~\ref{table:utf8scheme} shows the
translation of a \gls{codepoint} into a bit sequence when using
\glslink{UTF}{UTF-8}.

\begin{table}[H]
 \begin{center}
    \scalebox{0.7}{
      \begin{tabular}{crrccccccc}
        Bits of    & First      & Last       & Bytes in & Byte 1 & Byte 2 & Byte 3 & Byte 4 & Byte 5 & Byte 6 \\
        \gls{codepoint} & \gls{codepoint} & \gls{codepoint} & sequence &&&&&& \\ \hline
        7          & U+0000     & U+07FF     & 1        & 0xxxxxxx &&&&& \\
        11         & U+0080     & U+FFFF     & 2        & 110xxxxx & 10xxxxxx &&&& \\
        16         & U+0800     & U+FFFF     & 3        & 1110xxxx & 10xxxxxx & 10xxxxxx &&& \\
        21         & U+10000    & U+1FFFFF   & 4        & 11110xxx & 10xxxxxx & 10xxxxxx & 10xxxxxx && \\
        26         & U+200000   & U+3FFFFFF  & 5        & 111110xx & 10xxxxxx & 10xxxxxx & 10xxxxxx & 10xxxxxx & \\
        31         & U+4000000  & U+7FFFFFFF & 6        & 1111110x & 10xxxxxx & 10xxxxxx & 10xxxxxx & 10xxxxxx & 10xxxxxx \\
      \end{tabular}
    }
  \end{center}
  \caption{Scheme of \glslink{UTF}{UTF-8} encoding \cite{wikipediautf8table}}
  \label{table:utf8scheme}
\end{table}

With the help of Table~\ref{table:utf8scheme}, it is easy to convert a Unicode
\gls{codepoint} into \glslink{UTF}{UTF-8} binary data. To encode the Thai
character `no nen', the following has to be done: The Unicode \gls{codepoint}
is `U+0e13', which is `111000010011' in binary representation. `U+0e13' is
between `U+0800' and `U+FFFF'. Thus, we need three bytes to encode this
character. The result would be: `11100000 10111000 10010011'. The
\glslink{UTF}{UTF-16} encoding is a bit more complicated. However, when
encoding a character of \gls{plane} 0, the encoding is numerically equal to the
corresponding \gls{codepoint}. Thus, `U+0e13' is represented as `00001110
00010011' in a \glslink{Endianness}{big endian} system and as `00010011
00001110' in a \glslink{Endianness}{little endian} system.

\subsection{Unicode in modern software}

A Unicode enabled operating system should meet the following
criteria:
\begin{itemize}
  \item File names can contain Unicode characters.
  \item System software can handle Unicode in file names, command line
    parameters, etc.
  \item User applications, such as text editors, are able to support Unicode
    data.
\end{itemize}
Nowadays, nearly all modern operating systems support Unicode. Of course
there are many different implementations, depending of the operating system's
history and the developer's decisions. Linux uses \glslink{UTF}{UTF-8} as its
default internal data format, while Microsoft decided to store data with the
\glslink{UTF}{UTF-16} format.\cite{unicodeinlinux} To support legacy software
which cannot handle Unicode, an operating system can provide an \gls{API} to convert
between different encodings. If, for example, a Microsoft Windows application
has to handle legacy code page data, it can make use of the Windows API
functions MultiByteToWideChar \cite{multibytetowidechar} and
WideCharToMultiByte \cite{widechartomultibyte} to convert from Unicode to
legacy code pages and back. \\
However, this can lead to loss of information and should only be used when
absolutely necessary. \cite{codepages} To write internationalised software, it
is very important that the used programming language and its compiler can
handle Unicode encoded data. \\
Fortunately, most of those development tools have full support for this today.
Java and .NET use a \glslink{UTF}{UTF-16} representation for every string or
character object. Google's Go programming language works with
\glslink{UTF}{UTF-8} internally and finally, the core libraries of C
and C++ have added wide-string manipulation functions in addition to the legacy
`char*' functions. However, the internal data representation is compiler
dependent and can vary from system to system. Thus, the Unicode support is the
programmer's task in these two languages. \cite{unicodeproglang} However, there
are third party libraries, such as \gls{ICU}, which are adding native Unicode
support to C and C++.  \gls{ICU} also is part of the Java \gls{SDK} to support
Unicode for Java programmers in a transparent way. \cite{icu}

\section{Technical issues}

With or without Unicode, there are several problems for programmers and end
users. Even as Unicode appears as the globalised character encoding scheme, it
is not perfect. The use and implementation of Unicode can lead to unexpected
outcomes.

\subsection{Detecting a file encoding}

Detecting the encoding used for a file can be a difficult task. \gls{ASCII},
\gls{ISO-8859} and Windows code pages do not prepend \glspl{magicnumber}
for identification. \glslink{UTF}{UTF-8} and \glslink{UTF}{UTF-16} may or may
not include a byte order mask at the beginning of the document.  Therefore,
there is no pre-determined way of revealing the right encoding with only the
file as the information source. The only way to determine the encoding is to
guess, which might lead into unexpected representations of some characters. If
the file only contains characters in the \gls{ASCII} range (0-127), then
guessing may be an easy task. On the other hand, if there are characters with
the eighth bit set, there is no indication of which encoding to assume. If, for
example, the file contains `D0 AE', there are at least four different ways to
interpret it:

\begin{itemize}
  \item Assuming 8-bit \gls{ANSI} (Windows \glslink{cp-125x}{cp-1252}), there are two
    characters: `U+00D0' and `U+00AE', or ``Đ\textregistered''
  \item Assuming \glslink{UTF}{UTF-8}, there is only one character `U+042E', or
    ``{\droid Ю}''
  \item Assuming \glslink{UTF}{UTF-16} in \glslink{Endianness}{big endian}, it
    would be `U+D0AE', or ``{\droid 킮}''
  \item Assuming \glslink{UTF}{UTF-16} in \glslink{Endianness}{little endian},
    it would be `U+AED0', or ``{\droid 껐}''
\end{itemize}
All of these decisions could make sense, so there is no way to know which one
is correct. \cite{notepadproblem} \\
A real world example for a faulty detection algorithm is the `Bush Hid The
Facts'-Bug. This bug, which was often used as a hoax in online forums, can be
reproduces on every Windows XP or Windows NT/2000 version. A text file, created
with notepad.exe, containing only the text ``bush hid the facts'' will show
nonsense squares and Chinese characters after reopening. \cite{bushhidthefacts}

\subsection{Fonts}

A font is the essential software needed to render and display a character in an
appropriate way. If a font does not support the needed Unicode character set, it
is not able to present the characters to the user. It may chose the replacement
character `U+FFFD', if it has no \gls{glyph} available for the given \gls{codepoint}.
This special character can look like the following: {\mongolian �}
{\chineset￰�} {\arial￰�} {\dejavu￰�}.

\subsection{Gmail and a globalised world}

In 2012, the \gls{IETF} created a new standard for email addresses, to support
non-Latin and accented Latin characters. Google is the first email provider
who implemented this feature for its customers. \cite{gmailglobal} However,
being able to use such special characters in email addresses and domains, can
lead to a high security risk. The Myanmar letter Wa (`U+101D', or `{\myanmar
ဝ}'), the Gujarati digit zero (`U+AE6', or `{\gujarati ૦}') and the Greek small
letter omicron (`U+03BF', or `{\droid ο}') are all very similar to the plain
\gls{ASCII} letter `o'. It might not be a direct technical issue, but in a
world of \gls{spam} and \gls{scam}, it is important to understand how this can
be used for nefarious purposes. For example, getting an email from
`a-g{\myanmar ဝ}{\droid ο}d-friend@d{\gujarati ૦}main.com.au', instead of
`a-good-friend@domain.com.au', and responding to it with sensitive information
can result in numerous negative outcomes. \cite{gmailprotectglobal}

\subsection{This document}

To illustrate the shortcomings of Unicode, we will examine some of the
difficulties of compiling this document. It was created with vim \cite{vim}, a
Unicode-compatible text editor on a Gentoo Linux \cite{gentoo} using
en\_US.\glslink{UTF}{UTF-8} as its default locale. To compile the \LaTeX~
document, the XeLaTeX-compiler \cite{xelatex} was used, which has native
Unicode support. Typing a special Unicode character is easy. Just by pressing
<Ctrl+Shift+u>, followed by the \gls{codepoint}'s hex-code and <Enter>. The
plain text editor displayed every character correctly, but the compiled
PDF-document usually displayed only a blank, or a replacement character, for
the non-Latin symbols. This occurs, because the \LaTeX compiler assumes only
one predefined language per document, which is English in this case. There are
two common ways to resolve this problem. The `babel' package has support for
switching languages on the fly with help of special commands. The other simple
way is to define multiple fonts in the preamble of the document, which could be
used explicitly for special characters. This was the more reliable solution for
this document, because only single foreign characters were used.  For this task
the Noto font package from Google \cite{noto}, which has the goal to support
every Unicode character, was very helpful.

\section{Future of character encoding}

At first glance, Unicode looks like the solution to all character encoding
problems. But as we have seen, there are still many problems left to resolve.
Of course, the standard has to be maintained. New characters have to be added,
and obsolete ones have to be deleted. It also would be very nice if there was
one unique encoding scheme, which satisfies all needs and requirements. \\
It is very likely that more fonts, operating systems and user software will
support Unicode in the future, which definitely seems to be the right approach in
a globalised and localised software world. However, in the current situation it does
not look like there is an easy way to solve all the problems.

\section{Conclusion}

We have seen that humanity began encoding characters long before there were
any thoughts on globalisation or localisation of software. This simple looking
task gained more and more complexity in a world where computers and the
interaction between them and humans evolved into an indispensable task. The
development of Unicode at the beginning of this globalised market was a very
important step in the direction of software localisation without reinventing
basic encoding schemes for every new software. On the other hand, we have
noted that there are several technical and non-technical problems left.
Languages are complex and dynamic structures and can not easily be mapped into
a simple computer scheme. Although there is no need for another, better
encoding standard, Unicode has to be maintained and further developed to
satisfy the needs of all globalised and localised software development aspects.

\newpage
\printglossary[style=long3col]
%\listoffigures
\listoftables
%----------------------------------------------------------------------------------------
%BIBLIOGRAPHY
%----------------------------------------------------------------------------------------
\bibliographystyle{IEEEtran}
\begingroup
\raggedright
\bibliography{bibtex}
\endgroup
%----------------------------------------------------------------------------------------
%\printindex
\end{document}
